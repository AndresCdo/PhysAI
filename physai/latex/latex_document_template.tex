\documentclass[11pt]{article}
\usepackage{amsmath}
\usepackage{amsfonts}
\usepackage{amssymb}
\usepackage{graphicx}
\usepackage{hyperref}

% Set paper size and margins
\usepackage[a4paper, margin=1in]{geometry}

\title{PhysAI: Linking Quantum Mechanics and General Relativity}
\author{Andres Caicedo}
\date{\today}

\begin{document}

\maketitle

\begin{abstract}
    This document describes the methods, results, and contributions of the PhysAI project, an open-source initiative aimed at generating physical equations to link quantum mechanics and general relativity using artificial intelligence and machine learning algorithms. 
\end{abstract}

\section{Introduction}
    The main goal of the PhysAI project is to connect the equations of quantum mechanics and general relativity, two of the most fundamental physical theories describing the behavior of matter and energy in the universe.

\section{Background}
    \subsection{Quantum Mechanics}
        Briefly describe the key concepts and equations of quantum mechanics that are relevant to your project.
        
    \subsection{General Relativity}
        Briefly describe the key concepts and equations of general relativity that are relevant to your project.
        
    \subsection{Machine Learning and Artificial Intelligence}
        Briefly describe the machine learning algorithms and artificial intelligence techniques used in the PhysAI project.

\section{Methods}
    Describe the methods used in the PhysAI project, including data collection and preprocessing, machine learning algorithms, and equation verification.

\section{Results}
    Present the results of the PhysAI project, including any generated equations and their performance in explaining physical phenomena or linking quantum mechanics and general relativity.

\section{Discussion}
    Discuss the implications of your results, any limitations, and potential avenues for future research.

\section{Conclusion}
    Summarize the main contributions and findings of the PhysAI project.

\section*{Acknowledgements}
    Acknowledge any contributors, funding sources, or other support for the PhysAI project.

\begin{thebibliography}{9}
    % Include relevant citations
    \bibitem{citation1}
        Author1, Author2, Author3.
        \emph{Title of the article or book}.
        Journal or Publisher, Year.
\end{thebibliography}

\end{document}
